\documentclass{article}
\usepackage{amsmath, amssymb, amsfonts, framed}
\usepackage[left=1.5cm]{geometry}
\title{Coordinate Systems}
\date{Jan 29, 2023 - \today}
\begin{document}
\maketitle
\section{Introduction}
\begin{itemize}
    \item Relational Algebra is a basic set of operations for the relational model
    \item Expressions are compositions of relation algebra operations possible due to closure
    \item It's used as the model for SQL
\end{itemize}
\section{Select Operator}
\begin{itemize}
    \item Unary operator that returns subset of tuples from a relation given a selection condition
    \item Denote
        \begin{equation}
            \sigma_{\text{condition}}(R)
        \end{equation}
    \item In SQL we can express this as
        \begin{verbatim}
            SELECT *
            FROM R
            WHERE <condition>
        \end{verbatim}
    \item Selection cannot produce duplicates (relational model is set-based)
    \item
        \begin{equation}
            \sigma_{c_2}(\sigma_{c_1}(R)) = \sigma_{c_1}(\sigma_{c_2}(R))
    \end{equation}
    \begin{equation}
        \sigma_{c_2}(\sigma_{c_1}(R))  =\sigma_{c_1 AND c_2}(R)
    \end{equation}
\item Define the selectivity as the fraction of tuples selected by the selection condition
\end{itemize}
\section{Project Operator}
\begin{itemize}
    \item Unary operator that keeps specified attributes and discards others
    \item Denote
        \begin{equation}
            \pi_{\text{attributes}}(R)
        \end{equation}
    \item By nature, project returns a set of distinct tuples
    \item In SQL, we can express this as
        \begin{verbatim}
            SELECT DISTINCT <attributes>
            FROM R
        \end{verbatim}
    \item \textit{Note} $\pi_L(R)$ is only defined if $L\subseteq attr(R)$
    \item
        \begin{equation}
            \pi_{L_2}(\pi_{L_1}(R)) = \pi_{L_2}(R)
        \end{equation}
        \begin{equation}
            \pi_L(\sigma_C(R)) = \sigma_C(\pi_L(R))
        \end{equation}
    \item Define the degree as the number of attributes in projected attribute list
\end{itemize}
\section{Set Theory}
\begin{itemize}
    \item Many operators from set theory are also found in relational algebra
    \item Union and Intersection
    \item Difference $R-S$ returns the elements in R but not in S (complement of S union complement R)
\end{itemize}
\section{Cross Product}
\begin{itemize}
    \item Binary operator that returns all combinations of elements in A and B
    \item The resultant has degree equal to the sum of operand degrees and number of tuples equal to the product
    \item Relations do not have to be union compatible
\end{itemize}
\section{Renaming}
\begin{itemize}
    \item Unary operator that can rename relation, attributes or both
    \item Denote
        \begin{equation}
            \rho_{S(B_1,\dots,B_n)}(R)
        \end{equation}
    \item eg. Pairing upper years with F!rosh
        \begin{equation}
            \rho_{Mentor(senior, class)}(\sigma_{year>2}(Student))\times\sigma_{year = 1}(Student)
        \end{equation}
\end{itemize}
\section{Inner Join Operator}
\begin{itemize}
    \item Binary operator that crosses two relations and applies a selection condition
    \item Denote
        \begin{equation}
            R\bowtie_{condition}S = \sigma_{condition}(R\times S)
        \end{equation}
\end{itemize}
learning from https://cs.uwaterloo.ca/\~tozsu/courses/CS338/lectures/5\%20Rel\%20Algebra.pdf
\end{document}

