\documentclass{article}
\usepackage{amsmath, amssymb, amsfonts, framed}
\usepackage[left=1.5cm]{geometry}
\title{Legendre Polynomials}
\date{Jan 21, 2023 - \today}
\begin{document}
\maketitle
    \section{What is a Legendre Polynomial}
    \begin{itemize}
        \item Legendre polynomials of the first kind are also referred to as zonal harmonics
        \item They are part of the general solution to Legendre's differential equation
        \item The nth polynomial is of order $n$
            \begin{equation}
                P_n(x) \sim x^n
            \end{equation}
        \item An interesting property is that Legendre polynomials are orthogonal over the interval $I = [-1;1]$ and satisfy
            \begin{equation}
                \int_{-1}^{1} P_m(x)P_n(x) \mathrm{dx} = \frac{2}{2n+1}\delta_{mn}
            \end{equation}
        \item They also satisfy
            \begin{equation}
                P_n(1) = 1 \label{Eq:unitary}
            \end{equation}
    \end{itemize}
    \section{Generating Legendre Polynomials}
    \begin{itemize}
        \item Given the first two Legendre polynomials
            \begin{align}
                P_0(x) &= 1\\
                P_1(x) &= x
            \end{align}
            We can generate a sequence of polynomials by generating $n+1$ equations and solving for the coefficients using the inner product of known polynomials
    \end{itemize}
    \begin{leftbar}
        \subsection{$P_2$}
        \begin{itemize}
            \item We assume that $P_2$ has the form $ax^2+bx+c$
                \begin{align}
                    \int_{-1}^{1} (1)P_2 \mathrm{dx} &= \frac{ax^3}{3}+\frac{bx^2}{2}+cx\bigg|_{-1}^{1}\\
                    &= \frac{2a}{3}+2c = 0 \\
                    \int_{-1}^{1} (x)P_2 \mathrm{dx} &= \frac{ax^4}{4}+\frac{bx^3}{3}+\frac{cx^2}{2}\bigg|_{-1}^{1} \\
                    &= \frac{2b}{3} = 0 \implies b = 0
                \end{align}
            \item Using the property from Equation (\ref{Eq:unitary}) we also find
                \begin{equation}
                    a+c=1
                \end{equation}
        \end{itemize}
        Finally we arrive at
        \begin{equation}
            P_2(x) = \frac{1}{2}(3x^2-1)
        \end{equation}
    \end{leftbar}
    \begin{leftbar}
        \subsection{$P_3$}
        \begin{itemize}
            \item We can similarily find $P_3$
            \item Assuming the form $P_3 = ax^3 + bx^2 +cx + d$
                \begin{align}
                    \int_{_1}^{1} (1)P_3 \mathrm{dx} &= \frac{ax^4}{4}+\frac{bx^3}{3}+\frac{cx^2}{2}+dx\bigg|_{-1}^{1}\\
                                                     &= \frac{2b}{3}+2d = 0 \label{Eq:P3res1} \\
                    \int_{-1}^{1} (x)P_3 \mathrm{dx} &= \frac{ax^5}{5} + \frac{bx^4}{4}+\frac{cx^3}{3}+\frac{dx^2}{2}\bigg|_{-1}^{1}\\
                                                     &= \frac{2a}{5}+\frac{2c}{3}=0 \label{Eq:P3res2} \\
    \int_{-1}^{1} \frac{1}{2}(3x^2-1)P_3 \mathrm{dx} &= \frac{1}{2} \int_{-1}^{1} 3ax^5 + 3bx^4 + 3cx^3 + 3dx^2-ax^3-bx^2-cx-d\, \mathrm{dx} \\
                                                     &= \frac{3ax^6}{6}+\frac{3bx^5}{5}+\frac{(3c-a)x^4}{4} + \frac{(3d-b)x^3}{3}-\frac{cx^2}{2}-dx\big|_{-1}^{1} \\
                                                     &= \frac{6b}{5}+\frac{6d-2b}{3}-2d =0 \label{Eq:P3res3}
                \end{align}
            \item Again we can find
                \begin{equation}
                    a + b + c +d =1
                \end{equation}
            \item Note that Step (\ref{Eq:P3res3}) and by extension Step (\ref{Eq:P3res1}) yield
                \begin{equation}
                    b=0\quad d=0
                \end{equation}
            \item Step (\ref{Eq:P3res2}) and evaluating $P_3(1)$ then yield
                \begin{equation}
                    c = -\frac{3}{2}\quad a=\frac{5}{2}
                \end{equation}
        \end{itemize}
        We arrive at
        \begin{equation}
            P_3(x) = \frac{1}{2}(5x^3-3x)
        \end{equation}
    \end{leftbar}
    \end{document}
