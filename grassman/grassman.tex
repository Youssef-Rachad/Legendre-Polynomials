\documentclass{article}
\usepackage{../math}
\begin{document}
\section{Grassman numbers}
\begin{itemize}
    \item Are tired of integration? Do you enjoy derivatives instead? Boy do I have the number for you!!
    \item Grassman numbers are named after Hermann Grassman and used in quantum field theory to describe fermions
    \item The first property of Grassman numbers is anticommutivity under multiplication
        \begin{equation}
            \eta_1\eta_2 = -\eta_2\eta_1
        \end{equation}
    \item This has the interesting consequence that the square, and by extension any higher power, of a Grassman number is 0 since
        \begin{equation}
            \eta_1\eta_2 = -\eta_2\eta_1 = 0
        \end{equation}
    \item Thus we can only express functions of Grassman numbers in the form
        \begin{equation}
            f(\eta) = a_0 + a_1\eta \quad (a_0, a_1) \in\mathbb{C}
        \end{equation}
    \item The derivative of a Grassman function works as usual, with the added constraint that the operator must be adjacent to the matching variable by way of commutation
        \begin{equation}
            \frac{\mathrm{d}}{\mathrm{d\eta_1}}\eta_2\eta_1 = -\frac{\mathrm{d}}{\mathrm{d\eta_1}}\eta_1\eta_2 = -\eta_2
        \end{equation}
\end{itemize}
\end{document}
