
\documentclass{article}
\usepackage{../math}
\title{Coordinate Systems}
\date{Jan 21, 2023 - \today}
\begin{document}
\maketitle
\section{Conversion}
\begin{itemize}
    \item The rectangular coordinates are called invariant and can be used to reference other system
    \item Converting from Spherical to Rectangular (define rectangular in terms of spherical)
        \begin{align}
            x &= r \sin{(\theta)} \cos{(\phi)}\\
            y &= r \sin{(\theta)} \sin{(\phi)}\\
            z &= r \cos{(\theta)}
        \end{align}
    \item From Rectangular to Spherical
        \begin{align}
            r &= \sqrt{x^2+y^2+z^2} \\
            \theta &= \arctan{\left(\frac{\sqrt{x^2+y^2}}{z}\right)} \\
            \phi &= \arctan{\left(\frac{y}{x}\right)}
        \end{align}
    \item If we express a vector as a sum of unit vectors that describe the system, we can convert it by determining the appropriate conversion factors. This is done by taking dot products.
    \item We get
        \begin{align}
            \hat{\mathbf{r}} & = \sin{(\theta)}\cos{(\phi)}\hat{\mathbf{x}}+\sin{(\theta)}\sin{(\phi)}\hat{\mathbf{y}}+\cos{(\theta)}\hat{\mathbf{z}}\\
            \hat{\mathbf{\phi}} & = -\sin{(\phi)}\hat{\mathbf{x}} + \cos{(\phi)}\hat{\mathbf{y}}\\
            \hat{\mathbf{\theta}} & = \cos{(\theta)}\cos{(\phi)}\hat{\mathbf{x}}+\cos{(\theta)}\sin{(\phi)}-\sin{(\theta)}\hat{\mathbf{z}}\\
            \\
            \hat{\mathbf{x}} & = \sin{(\theta)}\cos{(\phi)}\hat{\mathbf{r}}+\cos{(\theta)}\cos{(\phi)}\hat{\mathbf{\theta}}-\sin{(\phi)}\hat{\mathbf{\phi}}\\
            \hat{\mathbf{y}} & = \sin{(\theta)}\sin{(\phi)}\hat{\mathbf{r}}+\cos{(\theta)}\sin{(\phi)}\hat{\mathbf{\theta}}+\cos{(\phi)}\hat{\mathbf{\phi}}\\
            \hat{\mathbf{z}} & = \cos{(\theta)}\hat{\mathbf{r}}-\sin{(\theta)}\hat{\mathbf{\theta}}\\
        \end{align}
    \item Thus we can geometrically deduce the Jacobian matrix which describes the transformation
    \item An interesting property of the euclidean norm of a vector is that it is similar in rectangular coordinates and spherical coordinates:
        \begin{equation}
            \lVert \mathbf{r} \rVert = \sqrt{r^2+\phi^2+z^2}
        \end{equation}
\end{itemize}
\section{Laplace Operator}
\begin{itemize}
    \item The Laplace Operator is a useful one when dealing with waves and is fundemental to the Schrödinger equation
    \item This section will attempt at converting
        \begin{equation}
            %\Delta = \frac{\partial^2\Phi}{\partial x^2}+ \frac{\partial^2\Phi}{\partial y^2}+ \frac{\partial^2\Phi}{\partial z^2}
            \Delta = \frac{\partial^2}{\partial x^2}+ \frac{\partial^2}{\partial y^2}+ \frac{\partial^2}{\partial z^2}
        \end{equation}
        To Cylindrical and Spherical coordinate representations by making heavy use of differentiation rules.
    \item This process is both tedious and rewarding - only time will tell if this is worth the effort.
\end{itemize}
\subsection{Cylindrical Coordinates}
\begin{itemize}
    \item We begin by establishing cylindrical partial derivatives using the chain rule
        \begin{align}
            \frac{\partial}{\partial x} &= \frac{\partial r}{\partial x}\frac{\partial}{\partial r} +
 \frac{\partial \phi}{\partial x}\frac{\partial}{\partial \phi} \\
            \frac{\partial}{\partial y} &= \frac{\partial r}{\partial y}\frac{\partial}{\partial r} +
 \frac{\partial \phi}{\partial y}\frac{\partial}{\partial \phi} \\
        \end{align}
    \item Then evaluating partial derivatives for $r$
        \begin{align}
            \frac{\partial r}{\partial x} &= \frac{\partial}{\partial x}\sqrt{x^2+y^2} = \cos{(\phi)} \\
            \frac{\partial r}{\partial y} &= \frac{\partial}{\partial y}\sqrt{x^2+y^2} = \sin{(\phi)} \\
        \end{align}
    \item And for $\phi$ using painful implicit differentiation
        \begin{align}
            \frac{\partial}{\partial x}\cos{(\phi)} &= \frac{\partial}{\partial x}\frac{x}{\sqrt{x^2+y^2}}\\
            -\sin{(\phi)}\frac{\partial \phi}{\partial x} &=  \\
            \frac{\partial}{\partial y}\sin{(\phi)} &= \frac{\partial}{\partial y}\frac{x}{\sqrt{x^2+y^2}}\\
            \cos{(\phi)}\frac{\partial \phi}{\partial y} &=  \\
        \end{align}
    \item Notice that the $z$ term remains unchanged due to the relationship between polar and cylindrical coordinates. We obtain for free the Laplace operator in polar coordinates by omitting the $z$ term (or conversly, we obtain the cylindrical operator for free by ignoring the $z$ as was essentially done).
\end{itemize}
\subsection{Spherical Coordinates}
\begin{itemize}
    \item More of the same but definitely the most brain racking
    \item Establishing spherical partials
        \begin{align}
            \frac{\partial}{\partial x} &= \frac{\partial r}{\partial x}\frac{\partial}{\partial r} +
 \frac{\partial \phi}{\partial x}\frac{\partial}{\partial \phi} +
\frac{\partial \theta}{\partial x}\frac{\partial}{\partial \theta}  \\
            \frac{\partial}{\partial y} &= \frac{\partial r}{\partial y}\frac{\partial}{\partial r} +
 \frac{\partial \phi}{\partial y}\frac{\partial}{\partial \phi} +
\frac{\partial \theta}{\partial y}\frac{\partial}{\partial \theta}  \\
            \frac{\partial}{\partial z} &= \frac{\partial r}{\partial z}\frac{\partial}{\partial r} +
 \frac{\partial \phi}{\partial z}\frac{\partial}{\partial \phi} +
\frac{\partial \theta}{\partial z}\frac{\partial}{\partial \theta}
        \end{align}
\end{itemize}
\end{document}
